\begin{resumo}

O objetivo deste trabalho é pesquisar, propor e desenvolver um sistema que
capacite um veículo terrestre a se locomover de modo autônomo em ambientes
externos não estruturados ou pouco estruturados, ou seja, em um campo com
vegetação/plantação e/ou em uma floresta pouco densa (\textit{outdoor} e
\textit{off-road}). O veículo deverá ser capaz de se dirigir até uma localização
pré-determinada (coordenada GPS) escolhendo por meios próprios o caminho a
seguir, ao mesmo tempo em que desvia de obstáculos, percebendo-os de forma
autônoma. O sistema de navegação autônoma irá se basear na aquisição e
processamento de imagens, obtidas a partir de um par de câmeras (câmera
estéreo), constituindo assim um sistema de visão binocular do qual é possível se
obter uma percepção tridimensional do ambiente. Portanto, pretende-se extrair
parâmetros de navegabilidade do ambiente percebido pela câmera estéreo, como
caminhos livres, obstruções e obstáculos, que combinados com as informações de
posição e orientação do veículo e do ponto de destino (baseando-se em
coordenadas de GPS) serão integrados em um sistema robusto de navegação.

% Este trabalho irá utilizar a plataforma CaRINA I do LRM-ICMC/USP e INCT-SEC
% dotado de uma câmera estéreo, GPS e bússola, e atuadores usados para o
% controle de tração e direcionamento do veículo.
Inicialmente serão estudados e trabalhados algoritmos para a criação do mapa de
disparidade a partir do par de imagens obtidas pela câmera estéreo.
% Para a criação deste mapa o algoritmo deverá ter um compromisso de desempenho
% entre a qualidade (disparidade/profundidade) e a performance em termos de
% tempo de processamento, uma vez que a aplicação final deverá executar em tempo
% real (\textit{soft real-time}).
A partir do mapa de disparidade, será elaborado um mapa local de navegabilidade
que irá processar e classificar o espaço tridimensional percebido, separando e
representando as regiões navegáveis (seguras) e as regiões não navegáveis
(obstáculos e regiões à evitar) do espaço em frente ao veículo. Este mapa local
será utilizado em conjunto com as informações de posição atual e de destino (GPS
e bússola) a fim de realizar o controle da navegação do veículo. Estão sendo
consideradas duas abordagens principais para o controle local da navegação: a
primeira baseada no uso de Redes Neurais Artificiais, conforme proposto em
trabalhos anteriores desenvolvidos por membros do grupo do LRM e a segunda
baseada em uma adaptação do algoritmo VFH. Nestas abordagens serão consideradas
como parâmetro de entrada as informações tridimensionais do mapa de
navegabilidade.
% Além disto, também serão necessários estudos que visam identificar, a partir
% das imagens da câmera estéreo, o plano de referência de base (chão), seus
% desníveis e obstáculos, classificando-os como elementos transponíveis ou não.
As principais contribuições esperadas deste trabalho são a adaptação e
aperfeiçoamento dos algoritmos para a geração de mapas de disparidade, a
proposta e o desenvolvimento de algoritmos para a obtenção de mapas locais de
navegabilidade com informações espaciais (3D), e por fim o aperfeiçoamento de
técnicas previamente desenvolvidas para detectar e desviar de obstáculos em
mapas 2D, a fim de permitir uma navegação baseada no mapa de navegabilidade 3D.

%Este trabalho resultará em um sistema com
%possibilidade de aplicação em importantes tarefas de navegação, como por
%exemplo, em sistemas voltados para aplicações agrícolas e em sistema de combate
%a incêndio em florestas.

\end{resumo}

% 2012-10-10 lido ok
