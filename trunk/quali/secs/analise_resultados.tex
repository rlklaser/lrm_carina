\section{Forma de Análise dos Resultados}

Os resultados serão analisados comparativamente com as soluções já desenvolvidas
no LRM e em comparação com projetos semelhantes e relacionados. Além disto,
serão avaliadas e comparadas as diferentes abordagens adotadas neste estudo,
como por exemplo, comparando a abordagem baseada em VFH com RNAs. Outro quesito
a ser avaliado é o desempenho geral do sistema, onde serão realizadas medições e
avaliações dos tempos de processamento e do desempenho alcançados.

%A adaptabilidade e robustez final do resultado para as áreas de interesse e aplicação citadas nesta
%proposta irá indicar o grau de sucesso atingido. As necessidades de melhorias e a delineação de
%avanços que devem ser alcançados poderão servir de base para novos projetos, constituindo um
%avanço na prospecção tecnológica para o desenvolvimento de um sistema eficiente e robusto de
%navegação autônoma.

A métrica quantitativa mais utilizada para este tipo de sistema são as
frequências em que se permite executar os algoritmos. No caso de imagens a taxa
é dada em FPS (imagens por segundo). Estas taxas tem como principal papel
expressar os resultados de forma que possam ser utilizados para verificar se é
possível construir um sistema tempo real. Uma outra métrica relevante para
categorizar os métodos é uma avaliação qualitativa de quão paralelizável é o
algoritmo utilizado. Esta métrica geralmente acaba sendo apresentada de forma
implícita na descrição da estrutura do algoritmo.

%Para análises estatísticas geralmente são necessários critérios  

%OSORIO
% A validação por simulação permite também gerar situações mais variadas e
% avaliar os limites e capacidades do sistema proposto: variando 
% a densidade/quantidade de obstáculos presentes no ambiente; avaliando a
% quantidade de colisões que podem vir a ocorrer em determinadas situações; 
% verificando em que situações um obstáculo pode vir a não ser detectado; e
% assim por diante.

A primeira etapa consistem em avaliar o sistema em ambiente simulado. A
validação por simulação permite gerar diversas situações desejadas para testar
variados comportamentos sem a necessidade de criá-las no ambiente real, o que
dependendo do cenário pode ser arriscado. A simulação permite maior controle
sobre o ambiente permitindo cenários mais propícios e menos propícios afim de
verificar a robustez do sistema. As análises em ambiente real serão comparadas
com os testes efetuados em simulação permitindo uma avaliação do desempenho,
tanto da qualidade das trajetórias geradas como a capacidade de desvio de
obstáculos.

%OSORIO
%> As "considerações finais" deste capítulo podem ser uma lista da principais
%contribuições esperadas deste trabalhos descritas sobre a forma de uma lista de
% itens.
