\section{Objetivo e Hipótese}

O objetivo deste trabalho é desenvolver um sistema de navegação autônoma baseado
em visão computacional a fim de capacitar um veículo terrestre a se locomover em
ambientes externos não estruturados, ou seja, um campo com vegetação/plantação
e/ou floresta pouco densa. O veículo deverá ser capaz de desviar de obstáculos,
percebendo-os de forma autônoma, e se dirigir até uma localização determinada
escolhendo por meios próprios o caminho a seguir. Deverá ter alguma capacidade
de reconhecer o terreno que irá se deslocar a fim de evitar zonas não
transponíveis ou muito acidentadas.

A informação espacial tridimensional representa mais precisamente o ambiente
onde o veículo autônomo se encontra e por onde deve se locomover, portanto, os
métodos de navegação e representação do ambiente baseados nesta informação
tridimensional são mais abrangentes e menos limitados em relação ao controle e
conhecimento do ambiente.

Tem-se por hipótese então, que algoritmos de navegação planares podem ser
expandidos para o espaço tridimensional se tornando menos limitados e mais
robustos.

%2012-10-15 Revisar