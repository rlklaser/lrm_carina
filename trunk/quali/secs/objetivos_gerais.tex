\section{Objetivo Geral}
\label{sec:obj_geral}

O objetivo geral deste trabalho é desenvolver um sistema de navegação autônoma baseado
em visão computacional a fim de capacitar um veículo terrestre autônomo a se locomover em
ambientes externos não estruturados, ou seja, em campos com vegetação/plantação e/ou em
florestas pouco densas. O veículo deverá ser capaz de se dirigir até uma localização determinada,
desviando dos obstáculos, percebendo-os de forma autônoma, e escolhendo por meios próprios o
caminho a seguir. O veículo autônomo deverá ter a capacidade de identificar os elementos do terreno
onde irá se deslocar, identificando o chão e os obstáculos, a fim de evitar zonas não transponíveis ou
muito acidentadas. Portanto, o objetivo deste trabalho é desenvolver um sistema de navegação
robusto e seguro voltado à aplicação em veículos autônomos para ambientes externos não
estruturados (ou muito pouco estruturados), baseado no uso de visão computacional realizada
através da captura de imagens estéreo, no uso de mapas de profundidade (mapas de disparidade), e
no uso mapas locais de navegabilidade.

%2010-10-10 REVISAR