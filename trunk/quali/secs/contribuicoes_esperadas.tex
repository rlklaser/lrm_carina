\section{Contribuições Esperadas}

As principais contribuições acadêmico-científicas esperadas deste trabalho são: 

\begin{enumerate}[i.]

\item adaptação e aperfeiçoamento dos algoritmos para a geração em “tempo real”
de mapas de disparidade, obtendo estes mapas a partir de um par de imagens
capturadas pela câmera estéreo;

\item proposta e desenvolvimento de algoritmos para a obtenção de mapas locais
de navegabilidade com informações espaciais (3D), onde o espaço tridimensional
será dividido em regiões e estas regiões serão identificadas como sendo
navegáveis ou não navegáveis;

\item aperfeiçoamento de técnicas para a navegação baseada no uso de GPS,
bússola e mapas locais de navegabilidade, onde as pesquisas previamente
desenvolvidas para detectar e desviar de obstáculos com o uso de mapas 2D, serão
estendidas a fim de trabalhar com mapa de navegabilidade/ocupação em 3D. Deste
trabalho resultará um sistema com possibilidade de aplicação prática em
importantes tarefas de navegação autônoma, como por exemplo, em sistemas
voltados para aplicações agrícolas e em sistema de combate a incêndio em
florestas, tarefas estas que podem ser perigosas para o ser humano (por exemplo,
exposição prolongada aos produtos químicos de defensivos agrícolas, e
combate/contato direto com fumaça e incêndios).

\end{enumerate}
