\section{Objetivo e Motivação}
%\label{sec:motivacao}

Este trabalho tem por objetivo o desenvolvimento de um sistema de navegação
autônoma baseado em visão computacional a fim de capacitar um veículo terrestre
autônomo a se locomover em ambientes externos não estruturados, ou seja, em
campos com vegetação/plantação e/ou em florestas pouco densas. O veículo deverá
ser capaz de se dirigir até uma localização determinada, desviando dos
obstáculos de forma autônoma, e escolhendo por meios próprios o
caminho a seguir. O veículo autônomo deverá ter a capacidade de identificar os
elementos do terreno onde irá se deslocar, identificando o chão e os obstáculos,
a fim de evitar zonas não transponíveis ou muito acidentadas. Este tipo de
sistema robótico tem vasta aplicabilidade, o interesse por essas aplicações na
área da robótica móvel tem sido demonstrado tanto pelo surgimento de grupos de
pesquisa como pelas conferências e publicações na área.

Exemplo de aplicação deste tipo de sistema pode ser visto no trabalho de
\cite{Pessin2008} onde foi avaliado a utilização de veículos autônomos no
combate a incêndios florestais. Este tipo de aplicação requer que um (ou mais)
veículo se desloque de um ponto de origem (base) até um destino determinado
(foco de incêndio) utilizando coordenadas de GPS desviando de obstáculos e se
deslocando em um terreno irregular e não estruturado.

Por outro lado, a visão computacional ainda é uma área com diversos problemas em
aberto (o próprio funcionamento da visão animal ainda é pouco conhecido), sendo
uma importante fonte de informação sensorial para a robótica móvel. A informação
visual, assim como para os animais, tem forte relação com a capacidade de
localização e locomoção autônoma, sendo de grande relevância para a robótica
móvel.

%OSORIO
%Podia adicionar aqui na motivação um parágrafo que apresenta exemplos
% "concretos" de possíveis aplicações do sistema que está sendo proposto:
%- O exemplo mais direto é o tema tratado pelo Pessin (mestrado e doutorado), do
% combate a incêndios em florestas que necessitam de veículos autônomos com esta 
%capacidade de navegação entre 2 coordenadas (atual e destino) e com desvio de
%obstáculos.
%- Um exemplo mais arrojado é o dos robôs des exploração interplanetária (Marte)
% que também tem executado funções similares e que representam o estado-da-arte 
%nesta área da robótica (onde estes não são completamente autônomos!)...


%2010-10-14 REVISAR
